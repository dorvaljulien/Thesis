\chapter{Perspectives}

In this thesis we introduced the \HubLem fragmentation model and applied to the dynamical evolution of substructured star clusters and the fate of their binary population. Interesting results have been obtained, though some assumptions were made. In this chapter, we review two of these assumptions: the absence of stellar evolution and the isolated nature of the cluster. We also present the outline of a comparison to observations and the possible inclusion of hydrodynamical processes.


\minitoc



\section{Tidal field}

In all our simulations, it was assumed the clusters were in isolation and no tidal field were applied. This allowed to study the mechanisms of violent relaxation and the erasure of substructure. However, in reality, star forming regions are shaped by the gravitational influence of their surroundings. We mentioned in section \ref{Sec:2_conclusion} that the galactic tidal field could prevent the collapse of the \HubLem fragmented configuration and scatter the clumps, injecting them in the galactic cluster mass function. The fate of these clumps is uncertain, as some will disperse through two-body interactions, and other will merge, depending on the geometry imposed by the tidal field. 

Numerical simulations will allow an exploration of the resulting clump mass function, which would be more directly comparable to the cluster mass function. NBODY6 has a built-in galactic tidal field module, which allows to model the tidal forces associated with the cluster orbiting a complex galactic potential, including a \cite{Miyamoto1975} potential, see \cite{Aarseth2003} for details. The tidal shock from a passing molecular cloud is also an option in NBODY6.

However, it is possible to go further in the inclusion of realistic tidal fields. \cite{Renaud2011} introduced a new version of NBODY6, \href{http://personal.ph.surrey.ac.uk/~fr0005/nbody6tt.php}{Nbody6tt}, that can take an arbitrary tidal tensor as an input and apply it to the evolution of a star cluster. Specifically, Nbody6tt allows to extract the tidal environment of a cluster from a large-scale galactic simulation to obtain a time-dependant, self-consistent tidal field. The influence of different galactic environment, such as tidal arms, on the cluster can then be evaluated. 

For example, \cite{Renaud2015b} reported the formation of massive clusters in their hydrodynamical simulation of a galaxy merger analog to the Antennas galaxies. Fig~\ref{Fig:7_renaud} show the merging of YMC fragments on large scales. This kind of event could be reproduced in Nbody6tt with a \HubLem configuration and the tidal data from the simulation. Given the complex tidal fields in this kind of environment, the evolution of YMCs starting from a substructured initial conditions could shed light of their formation and disruption processes.


\begin{figure}
\begin{center}
\includegraphics[width=0.7\textwidth]{Figures/7_renaudantennas.png}
\end{center}
\caption{Cluster formation in a hydrodynamical simulation of the Antennaes galaxies. The colormap indicates gas densities and newly formed stars are shown as black dots. Two epochs are shown, outlining the erasure of substructure. The figure was extracted from \cite{Renaud2015b} }
\label{Fig:7_renaud}
\end{figure} 



\section{Stellar evolution}





No stellar evolution effect were included in our simulations. While choosing our stellar maximum masses (tables \ref{Tab:2_models} and \ref{Tab:6_models}) and the physical duration of our simulations (sections \ref{Sec:3_Scaling} and \ref{Sec:6_models}), we assumed stellar evolution would not significantly impact the global dynamical evolution of the substructured fragmented configurations. However, these mass ranges do not reflect the extent of the stellar mass spectrum observed in some young clusters, an often cited upper limit on stellar masses is 150~$\Mo$ \citep{Oey2005}.

Would the evolution of such massive stars affect the internal dynamics of clumps in our models ? To answer this, we turn to stellar models. Fig~\ref{Fig:7_stellarlifetime_hurley} shows the time needed for a star to reach the Giant Branch, which can be taken as a minimum lifetime, as a function of its mass. This analytical model from \cite{Hurley2000} reaches up to $\sim$ 60~$\Mo$, which gives a minimum lifetime of 6 Myr, while in our least dense models, with an initial stellar density of 6 stars/$\pc^{-3}$, the clumps take 6 Myr to merge and erase the substructure.

However, if we include tidal fields, the clumps might survive far longer, and the death of these massive stars could have a significant impact on their structures, their mass segregation, and  on the overall clump mass function.

Moreover, more massive stars are observed in clusters, and the lifetime alone does not reflect the mass-loss these stars endure throughout their life. Fig~\ref{Fig:7_stellarlifetime_weidner} shows the evolution of mass for several massive stars, 50, 65, 85 and 120~$\Mo$, from the Geneva stellar model \cite{Schaller1992}. These stars not only disappear in less than 4-5 Myr, they also lose a non-negligible portion of their mass before their death.

Throughout this thesis, we assumed stellar evolution would not affect the dynamics of our simulated clusters. For isolated systems, and for the densities and mass ranges we chose, this is mostly true. Nevertheless, as we include tidal fields and an extended mass spectrum to reproduce observed objects, mass-loss and other evolutionary effects need to be taken into account. This is especially true for binary evolution. Mass loss would affect the binary parameters, and some binary systems found in our systems have short enough separations to be contact binaries had stellar radii and evolution be included. These objects are crucial for the formation of blue stragglers or black holes.

NBODY6 has a built-in stellar evolution module based on the analytical model by \cite{Hurley2000}, including wind-driven mass loss, radii evolution, supernova event and stellar remnants. It was not used in this work for simplicity, but should be used for further research.



\begin{figure}
    \centering
    \begin{subfigure}[b]{0.60\textwidth}
        \centering
        \includegraphics[width=0.85\textwidth]{Figures/7_stellarlifetime.png}
        \caption{Time to reach the giant branch vs stellar mass.}
        \label{Fig:7_stellarlifetime_hurley}
    \end{subfigure}
    \begin{subfigure}[b]{0.38\textwidth}
        \centering
        \includegraphics[width=\textwidth]{Figures/7_stellarlifetime_weidner.png}
        \caption{Mass loss for massive stars.}
        \label{Fig:7_stellarlifetime_weidner}
    \end{subfigure}
     \caption{(a) Time taken to reach the Base of the Giant Branch (BGB) as a function of initial stellar mass, for two metallicities, Z = 0.0001 and Z = 0.03. The figure was extracted from \cite{Hurley2000}. (b) Evolution of mass over time for several massive stars. The figure was extracted from \cite{Weidner2006}. }
     \label{Fig:7_stellarlifetime}
\end{figure}




\section{Anisotropic expansion}

The \HubLem expansion we used throughout this thesis was isotropic, the velocity field was expressed with $\bold{v} = \Hub_0 \bold{r}$ with \tHub a scalar value. As a result, the fragmented configurations are roughly spherical and the net systemic angular momentum is null. This is a key difference between the method we have developed and the fractal approach of \cite{Goodwin2004}. Angular momentum may be significant in young clusters such as R136 \citep{Henault-Brunet2012}. In a fractal model, the way the velocity field is built leaves a residual, global angular momentum whereas the Hubble approach starts off with strictly zero angular momentum.

 A net angular momentum could be introduced in a Hubble model, for instance by setting 
\begin{equation}
\bold{v} = \Hub_o \bold{r} + \bold{\Omega} \times \bold{r}
\end{equation} 
with $\bold{\Omega}$ a chosen angular velocity. One can actually go further and write in matrix form
\begin{equation}
\bold{v} = \hat{\bold{H}} \bold{r} = \left(\begin{matrix}
H_{x,x} & H_{x,y} & H_{x,z}\\
H_{y,x} & H_{y,y} & H_{y,z}\\
H_{z,x} & H_{z,y} & H_{z,z}\\
\end{matrix}\right) \bold{r},
\end{equation}
with $\hat{\bold{H}}$ now a 3$\times$3 matrix, where the off-diagonal elements account for rotation and the elements on the diagonal $\Hub_{ii}$ control the three dimensional expansion. In this study, we have set $\Hub_{ij,i\ne j}=0$ and $\Hub_{ii}=\Hub_o$ otherwise. It is then a simple matter to study the fragmentation along a filament by setting, for example, $\Hub_{xx}=\Hub_{yy} < \Hub_{zz}$. 


\begin{figure}
    \centering
    \begin{subfigure}[b]{\textwidth}
        \centering
        \includegraphics[width=0.85\textwidth]{Figures/7_anisotropic.png}
        \caption{Anistropic \HubLem expansion.}
        \label{Fig:7_anistropic}
    \end{subfigure}
    
    \begin{subfigure}[b]{0.6\textwidth}
        \centering
        \includegraphics[width=\textwidth]{Figures/7_carina.png}
        \caption{Stellar density in Carina.}
        \label{Fig:7_carina}
    \end{subfigure}
     \caption{(a) Evolution of an anisotropic \HubLem model. The expansion was favored along z and a rotation was introduced. (b) Stellar density in the young Carina cluster the colorbar is in stars/$\pc^3$. The figure was extracted from \protect\cite{Kuhn2014}.}
     \label{Fig:7_filament}
\end{figure}




For example, to favor fragmentation along the z-axis and introduce a rotation along the same axis, the matrix can take the form
\begin{equation}
\bold{v} = \left(\begin{matrix}
0.2 + \cos \theta &  -\sin \theta & 0\\
\sin \theta & 0.2 + \cos \theta & 0 \\
0  & 0 & 1.7\\
\end{matrix}\right) \bold{r},
\end{equation}

with $\theta$ setting the orientation and strength of the rotation. We set $\theta = \frac{\pi}{4}$ and show the evolution of the resulting system with N=3000 on Fig~\ref{Fig:7_anistropic}. We obtain an elongated substructured configuration, that is comparable to the observed structure of, e.g., the Carina Nebula as observed in the MYStIX survey \citep{Kuhn2014} shown on Fig~\ref{Fig:7_carina}.



\section{Reproduce observations}






