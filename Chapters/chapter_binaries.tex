
\chapter{Binary population evolution} 
\label{Chapterbinaries}

\newpage
\section{Shock}


The equation 8.76 of \cite{BT} gives us the change of energy per unit mass of a self-gravitating system crossing a dense, planar region, in the context of a cluster crossing the galactic disk:

\begin{equation}
\label{Eq:3_DeltaEs}
\Delta E_s = \frac{14 \pi^2 G^2 \Sigma^2_d a^2}{3 V_z^2}
\end{equation}

with $\Sigma_d$ the surface density of the disk, $a$ the semi-major axis of stars orbits in the system and $V_z$ the incoming vertical velocity of the system on the disk. We apply this to a binary falling on the collapsed center of the system ($R_c$ being the half-mass radius at the point of deepest collapse), we get:
\begin{equation}
\label{Eq:3_sigmad}
\Sigma_d \simeq R_c \rho_c = R_c \frac{0.5 M}{\frac{4}{3} \pi R^3_c} = \frac{3 M}{8 \pi R_c^2}.
\end{equation}

The incoming velocity of a given binary can be approximated by the free fall velocity from the initial half-mass radius of the system $R_0$:

\begin{equation}
\label{Eq:3_Vz}
V_z^2 = \frac{G M}{R_0}
\end{equation}

Injecting equations \ref{Eq:3_sigmad} and \ref{Eq:3_Vz} into the expression \ref{Eq:3_DeltaEs} then multiplying by the mass of the binary gives the total energy change of the binary:

\begin{align}
\Delta E &= \frac{14 \pi^2}{3} \left( \frac{3}{8 \pi}\right)^2 G M m_{bin} a^2 \frac{R_0}{R_c^4}\\
 &= 0.65   \frac{G M m_{bin} a^2 }{C^4 R_0^3} 
\end{align}

with $C = \frac{R_c}{R_0}$ the concentration parameter. To get the ratio of energy change, we divide by the internal energy of the binary:

\begin{align}
\frac{\Delta E}{E} &= \frac{\Delta E}{G m_{bin}^2/a}\\
	&= \frac{0.65}{C^4} \cdot \frac{M}{m_{bin}} \cdot \left( \frac{a}{R_0} \right) ^3 
\end{align}

By making reasonable assumptions such as $C \propto N^{-\frac{1}{3}}$, $ M \propto N$ and $R_0 \propto N^{\frac{1}{3}}$ as we scale the system to preserve the initial density, it comes:

\begin{equation}
\frac{\Delta E}{E} \propto m_{bin}^{-1} ~ a^3 ~ N^{\frac{4}{3}} 
\end{equation}

Meaning that an increase of the number of particles will cause shorter binaries to suffer significant energy change. A factor 10 on membership reaches with axis shorter by a factor $10^{\frac{4}{9}} \simeq 2.8$.