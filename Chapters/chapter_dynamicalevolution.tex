


\chapter{Dynamical evolution} 
\label{ChapterDE}


\section{Bla}


%\begin{figure}
%\begin{center}
%\includegraphics[width=\columnwidth]{figures/Rhm_global_2}
%\caption{Half-mass radius as function of time for two systems undergoing collapse~: a uniform-density sphere (thin red solid curve) and a clumpy Hubble model (thick blue solid curve). Half-mass radii are in H.u, as well as the time axis, where $t_{Henon} = 1 {\rm unit} =  0.270 \times 10^5 $ years. Dashed lines are the half-mass radii of the same systems for the same systems but including only the bound stars.}
%\label{Fig:Rhm_global}
%\end{center}
%\end{figure}
%

%
\begin{table}
\begin{center}
\caption{Summary of collapse simulations and their characteristics. These simulations started from a subvirial state: cold uniform sphere or fully fragmented Hubble model; each were evolved up to t = 40 H.u}
\label{Tab:2_evolution_models}
\begin{tabularx}{\linewidth}{XXlXX}
\hline
Name & N & Mass range & Model \\
\hline
Rh100 & 15000 & [0.3 - 100] & Hubble  \\
Rh20 & 15000 & [0.35- 20 ] & Hubble  \\
Ru100 & 15000 & [0.3 - 100] & Uniform  \\
Ru20 & 15000 & [0.35- 20 ] & Uniform  \\
\hline
\end{tabularx}
\end{center}
\end{table}



The Hubble expansion comes to a halt at a time $\tau$ when $\theta(\tau) = \pi/2$  in Eq. (\ref{Eq:1_Hubble}). The system as a whole is then in a sub-virial state. We wish to explore briefly the  violent relaxation that follows and the equilibrium that ensues. In the present section, simulations will use the fully fragmented state of Hubble models as initial conditions for the subsequent dynamical evolution. Observationnal clues point to  collapsing and violently relaxing clusters. For example, \cite{Cottaar15} find IC348, a young (2-6 Myr) cluster, to be both super-virial and with a convergent velocity field, consistent with infalling motion. Dry collapse with no gas is an idealized situation: clearly if there was residual gas between the clumps and it was evacuated through stellar feedback, both the clump merger rate and the depth of the potential achieved during relaxation would be affected. As a limiting case, rapid gas removal may lead to total dissolution (see for instance \citealt{Moeckel12} and \citealt{Fujii16}). In the current situation, all clumps will merge.

The numerical integration were done once more with the Nbody6 integrator with the same computational units. For comparison purposes, we also performed simulations of cold uniform sphere, a configuration which has been extensively used  in the past (e.g.,\citealt{Theis99,Boily02,Barnes09,Caputo14,Benhaiem15}) and one that minimises the level of fragmentation and mass segregation in the on-set of collapse. Table~\ref{Tab:2_evolution_models} lists the simulations performed. We focus here on models with a mass function from 0.35$M_\odot$ to 20$M_\odot$ and 15000 stars, a compromise value for rich open clusters that should allow us to identify clearly collisional effects and trends with time, and ease comparison with the recent study by Caputo et al. (2014) where most calculations are performed with that sampling. We let both Hubble-fragmented- and uniform sphere evolve up to 40 H.u. 

\subsection{Scaling to physical units}
Before discussing the results, it is useful to translate the units of computation to physical scales. This is important if we want to discuss the state of the systems using one and the same physical time, such that the hypothesis of no stellar evolution holds good.
%  In H\'enon units, the free-fall time of a uniform density sphere is nearly a 
% rational fraction of the crossing time, $t_{ff} \simeq \sqrt{2}/3 t_{cr} = 0.4714 .. t_{cr}$, when the crossing 
% time is computed in equilibrium. 
This complicates the analysis, because all the time-scales defined in \S 3.1 are based on the hypothesis of equilibrium, and we start-off out of equilibrium. To make things clearer, let us resize the configurations so that the half-mass radius $r_h = 1$ pc initially, with a total system mass of $M = 15\times 10^3\, M_\odot$ for a volume density $\rho = M / (8\pi/3 r_h^3) \simeq 10^3 M_\odot / pc^3$, well within values typically inferred from observations. 
We then compute a free-fall time for the uniform-density model of $t_{ff} = \pi /2 \left( r_h^3 / GM \right)^{1/2} \approx (\pi/24)^{1/2} \times 10^5 $ years~; the {\it initial} crossing time would then be $t_{cr} = 2 r_h / \sigma_{1d} = 4\sqrt{6}/\pi \,t_{ff} $ where we invoked the virial theorem to compute $\sigma^2_{1d} = 1/3 \times GM/2 / r_h $. (The crossing time is larger than $t_{ff}$ because all the mass goes into the origin during free-fall.)  
In practice the more useful crossing time has to be computed from the equilibrium state achieved. If we once more invoke the virial theorem and note that at constant mass and energy the equilibrium state would reach a size $r_h^{eq} \approx r_h^{(0)}/2$ or half the initial radius, then one computes $t_{cr}^{eq} \approx t_{cr}^{(0)} / 2\sqrt{2} =  2\sqrt{3}/\pi t_{ff} \approx 1.10 \, t_{ff}$. 
Direct computation of the problem of a collapsing sphere give $t_{ff} \approx 1.36$ H.u. We therefore use the conversion factor \[ \frac{t}{{10^5\,{\rm years}}} = (4/3) / (\pi/24)^{1/2} \simeq  3.7\, t_{Henon} \,. \]  The equilibrium crossing time is then $t_{cr}^{eq} = 1.1\, t_{ff} = 3/2 \, t_{Henon} \approx 0.4 \times 10^5 $ years. The time-conversion factor adopted is conservative and does not factor in the stars that may escape during virialisation.%~: this would reduce the size of the bound cluster and boost the velocities of its stars. 
Thus by running up to 40 H.u we ensure that the systems evolve for at least 25 crossing times and $ \sim 10^6 $ years, short before the lifetime of massive stars.  With $N = 15 000$ and a mass range of $m_{max}/\langle m\rangle = 20$ we find from (\ref{Eqn:Trel}) and (\ref{Eqn:Tms}) a two-body relaxation time of $t_{rel} \approx 80\,t_{cr} $ (120 $t_{Henon}$, or 3 Myrs) and mass-segregation timescale of $t_{ms} \approx 4\, t_{cr} $ ($6\, t_{Henon}$, or $1.6\times 10^5$ years).  
  
\subsection{Collapse and virialisation}
\label{Subsec:Collapse}
The constant diffusion of kinetic energy by two-body interaction means that no stellar system ever reaches a steady equilibrium. However we can contrast the time-evolution of two configurations and draw conclusions about their observable properties. 

\begin{table}
\caption{Number of initially ejected stars in two collapse calculations} \label{tab:Ejectedstars}
\begin{tabular}{lrr}
Run & Ejected stars & Ejected mass  \\
\hline
Ru20  &  4227 & 27\% \\
Rh20  &  1932 & 12\% \\
\hline
\end{tabular}
\end{table}


With this in mind  we turn to Fig.~\ref{Fig:Rhm_global} in which we show the evolution of the half-mass radius for the cold uniform model (labeled Ru20~; thick red curve), and the Hubble model (labeled Rh20~; thin blue curve). Both systems have the same bounding radius initially, contract to a small radius when $t \simeq 1.4 $ units and then rebound at time $t \simeq 2 $ units. When all the stars are included in the calculation for $r_h$, we find that the radius increases at near-constant speed after the collapse. That trend does not appear to be slowing down which indicates that a fraction of the stars are escaping. The first batch of escapers is driven by the violent relaxation, however the trend continues beyond $ t = 10$ units, corresponding to $t > t_{ms}$ which implies two-body scattering and effective energy exchange between the stars. Note how the uniform model has a much deeper collapse and rebounds much more violently, shedding a fraction twice as large of its stars (Table~\ref{tab:Ejectedstars}). The half-mass radius $r_h$ increases steadily in both models, from the bounce at $t \approx 2$, until the end of the simulation (values in H.u):  

\begin{tabular}{lllrr}
% & Deepest (t=2) & t=40  \\
%\hline
$r_h$ Uniform & 0.11 &  $\rightarrow$ & 0.63 & ($\times 5$); \\
$r_h$ Hubble & 0.34  &  $\rightarrow$ & 0.49 & ($\times 1.4$). \\
%\hline
\end{tabular}\\

%
% from $ r_h \sim 0.11$ pc immediately after the collapse, to $r_h \sim 0.63$ pc at the end of the run, or a multiplicative factor $\approx 5$. In contrast, the Hubble run drops to $r_h \approx 0.34$ pc and rises over time to $r_h \approx  0.49$ pc (factor of $\approx 1.4$). 
Clearly the gentler collapse of the fragmented model has led to a more extended post-collapse configuration and reduced two-body evolution. Observe how the uniform model Ru20 is ejecting more stars than the Hubble model~: if we repeat the calculation for the Hubble run Rh20 but now include only the bound stars\footnote{See Appendix A for details of the selection procedure.}, the curve of $r_h$ obtained and shown as dash is shifted down but keeps essentially the same slope $\approx  0.004$. By contrast, the calculation for the bound stars of run Ru20 yields a much shallower slope than for the whole system: the slope drops from 0.015 to about 0.007. Irrespective of how the half-mass radius is calculated, the conclusion remains the same and agrees overall with the remark by \cite{caputo14} that boosting the kinetic energy of the collapsing initial configuration softens the collapse~; this was shown in a different context by \cite{theis99} and confirms these older findings.  Here, the fragmented model has finite kinetic energy due to the clumps' internal motion. The important new feature brought by the fragmented initial conditions is that the {\it mass profile} of the virialised configuration evolves much less over time in comparison. 

%%%%%%%%%%%%%%%%%%%
%
%
%\begin{figure*}
%\begin{center}
%\includegraphics[width=0.75\textwidth,clip=true]{figures/Lagr_radius_3_cropped}
%\caption{The ten-percentile mass radii (10\% to 90\%) as function of time. Radii and time axis are in H.u, with $t_{Henon} = 1 {\rm unit} =  0.270 \times 10^5 $ years. Left panels show the Uniform model and right panels show the Hubble fragmented models. Panels a and b show the evolution of the whole systems, while panels c and d shows the same radii computed for the bound stars only. Panel e shows the Uniform bound model (Ru20b) for which radius and time were rescaled to compensate the difference of initial kinetic energy (see text for details). Panel f shows the same information as panel d with smoothed data. 10\% and 90\% radii are labelled in the top right panel.
%%For both, the bottom panel show the bounded systems, with "b" appended to the name,  from which the ejected stars from the initial collapse were removed.
%}
%\label{Fig:Lagrange}
%\end{center}
%\end{figure*}


At the bounce, the half-mass radius of the Hubble model is $\approx 4$ times larger than that of the of the initially uniform sphere at rest (Fig. 10). The half-mass radii overlap at time $t \approx 15 \,H.u.$ (solid curves) or $t \approx 50 \,H.u.$ (dashed curves). Is the same trend applicable to all Lagrangian radii? To answer this question we plot on Fig. 11 the ten-percentile mass radii for the two models. The results are displayed for the two situations including all the stars (top row) or bound stars only (middle row). It is striking that the curves show very little evolution at all mass fractions for the case of the Hubble model (see right-hand panels on the figure), whereas all mass shells either contract or expand in time for the uniform one. We have noted how this model undergoes two-body relaxation on a timescale of $t \approx 10 \,H.u$: the innermost 10\% mass shell shows an  indication of \textit{core-collapse} at $t \simeq 5 \,H.u.$. We note here that the two sets of curves reach very similar values at the end of the calculations ($t = 40\, H.u$). A key difference between the two models, therefore, is that the final configuration of the Hubble model is almost identical to what it was at the bounce ; the same simply does not hold in the case of a uniform-density collapse. Furthermore, the Hubble calculation shows no hint of two-body relaxation or core-collapse. This raises the possibility that the system properties in the final configuration remain better correlated with those at the on-set of (global) collapse (we return to this point in \S7).

\cite{caputo14} and \cite{theis99} noted how a finite amount of kinetic energy in the {\it initial} configuration alters the  depth of the bounce during collapse. The ratio of half-mass radius at the bounce, to its initial value, is then $ r_h/r_h(0) \simeq Q_o + N^{-1/3}$, where $Q_o$ is the virial ratio of the initial configuration (see Fig. 5 of Caputo et al. 2014). We computed the kinetic energy of the  Hubble configuraiton and found that the internal motion of the clumps means that $Q_o (Hubble) \simeq 0.02$ for a Salpeter mass function with upper truncation value of $20 M_\odot$. With $N = 15k$ stars, the ratio $r_h/r_h(0) \simeq 0.041$ when $Q_o = 0$ shifts to $r_h/r_h(0) \simeq 0.061$ when $Q_o = 0.02$, or a factor close to 3/2. To account for the difference in kinetic energy of the initial configurations, we may therefore rescale the uniform model such that positions are  $ \times 3/2$ and the time unit is $\times (3/2)^{3/2} \simeq 1.84$. The new configuration would evolve in time in exactly the same way after mapping positions and time to their rescaled values. The result is shown as the bottom row on Fig. 11.  Note that we  have blown up the vertical axis to ease comparison between uniform- and Hubble models with bound stars only included. The rescaled uniform model is now slighlty more extended than before, but overal the final two configurations (at $t = 40 \,H.u.$) are as close as before rescaling. This demonstrates that  the outcome of the uniform collapse and its comparison with the Hubble model is not sensitive to a small amount of initial kinetic energy. We note that while the ratio $Q_o$ is a free parameter in many setups for collapse calculations, that parameter is fixed internally in the Hubble approach. 










\section{Global mass segregation}
\label{Sec:Segregation} 


To investigate the state of mass segregation in our models, we follow the analysis of Caputo, de Vries \& Portegies Zwart (2014). The masses are sorted by decreasing values, then subdivided into ten equal-mass bins. This means that the first bin contains the most massive stars. The number of stars in each bin increases as we shift to the following bins, since their mean mass {\it de}creases, and so on until we have binned all the stars. The half-mass radius $r_h$ computed for each bin is then plotted as function of time. In this way the mass segregation unfolds over time: if the stars were not segregated by mass, all radii $r_h$ would overlap. If two sub-populations share the same spatial distribution, their respective $r_h$ will overlap.

%
%\begin{figure*}
%\begin{center}
%\includegraphics[width=0.75\textwidth,clip=true]{figures/Rhm_segr_3_cropped}
%\caption{Half-mass radii of stars selected by mass as function of time. Each bin identified with 0-10\%, 10-20\% .. 90-100\%, contains ten percent of the total system mass. The stars where sorted by mass in decreasing order, and used to fill each ten-percent mass bin in order. Hence the first ten-percentile contains the most massive stars, the next ten-percentile the second group of massive stars, and so on until the 90-percent bin which contains the least massive stars in the model and is the most populated. Half-mass radius and time are in H.u, with $t_{Henon} = 1\,unit = 0.270 \times 10^5$ years. Left panels show the evolution of the Uniform model (Ru20, Ru20b) and right panels do the same for the Hubble model (Rh20, Rh20b). The organization of panels follows the same layout than figure~\ref{Fig:Lagrange} with a different factor for the rescaling of the uniform system. }
%\label{Fig:Rhm_segr}
%\end{center}
%\end{figure*}

Figure \ref{Fig:Rhm_segr} graphs the results for initially uniform-density- and fragmented Hubble models. The layout of the figure is the same as for Fig. 11. The violent relaxation phase leads to mass loss for both models and the much more rapid expansion of the half-mass radii of low-mass stars is an indication that most escapers have a lower value of mass. Fig.~\ref{Fig:Rhm_segr}(c) and (d) graphs $r_h$ for the bound stars of each sub-population. Clearly the initially uniform-density model is more compact early on, but note how the heavy stars sink rapidly to the centre, more so than for the case of the Hubble model. The spread of half-mass radii increases with time for both models, however two-body relaxation in the uniform-collapse calculation is much stronger, so that by the end of the simulations the half-mass radii of the low-mass stars of the respective models are essentially identical. Since the low-mass stars carry the bulk of the mass, that means that the two models achieve the same or similar mean surface density by the end of the run. At that time, the heavy stars in the uniform-collapse calculation are clearly more concentrated than in the Hubble run (compare the radii out to $\sim 40\%$ most massive stars). A direct consequence of this is that the {\it color} gradients of the core region of a cluster are much reduced when the assembly history proceeds hierarchically, in comparison with the monolithic collapse. It will be interesting and possibly important in future to compare such models with actual data for young clusters.

Another interesting remark is that the kinematics of the stars within the {\it system} half-mass radius is much different between the two models. For the Hubble calculation, the system half-mass radius, $ \approx 0.43 $ H.u, at $t = 40$ (cf. Fig. 11[d]) coincides with the half-mass radius of the $30-40\%$ bin stellar sub-population. All bins up to that range show little or no time-evolution, around the end of the run, which we interpret as efficient retention of these stars by the relaxed cluster. In the case of the uniform-collapse run, the system half-mass radius reaches $\approx 0.33$ H.u., which is significantly larger than the radius for the $30-40\%$ stellar sub-population. For that model, only the bins $0-10\%$ and $10-20\%$ are flat, and all the others increase almost linearly with time. Thus a fair fraction of bright stars deep in the cluster show systematic {\it outward streaming} motion, along with low-mass ones. This brings up the possibility to measure this signature motion through relatively bright stars, originating well inside the cluster half-mass radius. Recall that only post-bounce bound stars where selected to compute $r_h$ on Fig.~\ref{Fig:Rhm_segr}(c) and (d) ; the expansion is therefore not driven by chance events (e.g., Fig.~\ref{Fig:Rhm_segr}[a]), but rather through two-body relaxation. On the down side the bright tracers would be short-lived, and this may prove a strong constraint for observational detection.


Given the early dynamical evolution associated with substructured stellar clusters, some observed dense objects may yet be out of equilibrium (see \S\ref{Sec:discussion}). We wish to investigate the out-of-equilibrium state of our models just after the collapse. To ease the comparison between the two systems, the same rescaling procedure as for Fig~\ref{Fig:Lagrange} was applied to the uniform model, only this time the scaling was chosen so that the two clusters have comparable densities after the bounce. Lengths were multiplied by $4$; the time-axis is then scaled up by a factor $(4)^{3/2} = 8$. The result can be seen in panel (e); panel (f) shows a smoothed and zoomed in Hubble model for comparison.



\begin{table*}
\caption{Values of half-mass radii and their ratio to that of the most massive stars. The results are for the rescaled bound uniform model (rescaled Ru20b) and the bound Hubble model (Rh20b), after the collapse, and before dynamical mass segregation sets in.} \label{Tab:RhmVal}
\begin{tabular}{l|llllllllll}
Uniform (rescaled) & 0-10\% & 10-20\% & 20-30\% & 30-40\% & 40-50\% & 50-60\% & 60-70\% & 70-80\% & 80-90\% & 90-100\% \\
\hline
Radius   & 0.20 & 0.245 & 0.282 & 0.273 & 0.294 & 0.325 & 0.326 &  0.328 & 0.335 & 0.340 \\
Ratio    & 1.0 & 1.23 & 1.41 & 1.37 & 1.47  & 1.63 & 1.63 &  1.64 & 1.68 & 1.70 \\

\hline
Hubble  & 0-10\% & 10-20\% & 20-30\% & 30-40\% & 40-50\% & 50-60\% & 60-70\% & 70-80\% & 80-90\% & 90-100\% \\
\hline
Radius  &  0.18 & 0.21 & 0.286 & 0.293 & 0.316 & 0.321 & 0.333 & 0.338 & 0.342 & 0.344 \\
 Ratio       & 1.00 & 1.16 & 1.58 & 1.63 & 1.76  & 1.78  & 1.85 &  1.88 & 1.90 &  1.91 \\
\end{tabular}
\end{table*}



We compare the values of the different half-mass radii of the various population before the dynamical mass segregation sets in. This process is clearly visible as the drop of the half-mass radius of the most massive stars during the evolution. We are interested in the segregation which originates from the collapse and is present before this dynamical evolution. Table~\ref{Tab:RhmVal} sums up the values of the half-mass radii taken at $t\sim5$ for both models, both corresponding to the same unevolved post-collapse state (see arrows on panels [e] and [f] on Fig~\ref{Fig:Rhm_segr}). With on the order of $\sim 100$ stars per bin or more, one estimates roughly a ten-percent standard deviation from random sampling. To measure the \textit{relative} segregation between populations, the table also lists the ratios of each half-mass radius to the one for the most massive stars. Both models appear mass segregated (since these ratios are significantly greater than unity). The Hubble model is more segregated, on the whole, albeit in a different way compared to the uniform model. The segregation in that one is more regular and spreads over more mass bins. In the Hubble model, the segregation is much enhanced for the first two mass bins. Such differences in the degree and nature of segregation can be explained by the clumps structure before the collapse. We showed in \S\ref{Subsec:ClumpSegregation} the clumps were mass segregated with the most massive members being preferentially located at their center. The low membership and mass of most clumps implies that segregation mostly affects the very top of the stellar mass function. This segregation, predominant among massive stars, is then found in the resulting centrally concentrated system, after the collapse, and visible on Fig.~\ref{Fig:Rhm_segr}. The inheritance of mass segregation was studied by \cite{mcmillan07} for the case of merging Plummer spheres. \cite{allison10} furthermore showed that mass segregation in the system as a whole is enhanced for more filamentary  fractal initial condition (lower dimension, $D$ ; see their Fig. 5). Here our results confirm this observation. Mass segregation is a sensitive function of the initial clumpiness of the system and has immediate bearing on the dynamics of the virialised configuration, since all massive stars are more concentrated in the core.
