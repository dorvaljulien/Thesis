
\chapter*{Objective and structure of this thesis}

In this work, I present a new model for N-body simulations of young substructured star clusters, the \HubLem fragmentation. This model is based on an adiabatic expansion and fragmentation of an homogeneous system, which spontaneously develop clumps from initial overdensities. I will show this model recovers characteristics from hydrodynamical simulations of star formation, which are much more computationally expansive. The structure of the \HubLem model will be investigated, then applied to the study of the relaxation of young substructured clusters relaxation, as well as the evolution of binary populations in the same objects. 

\paragraph*{}
The thesis is organised as follow. An historical foreword recalls the intellectual and technological development that led to the current state of numerical astrophysics. An introduction presents the scientific context of the thesis, and the motivations for a new model. 

\paragraph*{}
The first part, subdivided in three chapter, introduces the \HubLem model itself, first with an analytical approach, then from the numerical point of view. The structural aspects of the model are investigated and compared to hydrodynamical simulations. Then, the fragmented system is used as initial conditions to study the violent relaxation of substructured cluster, comparing it to the collapse of uniform cold models.

\paragraph*{}
The second part, in two chapters, focuses on binary populations. A new binary detection algorithm is presented and its free parameter is calibrated. The spontaneous binaries arising during the \HubLem expansion are characterised, then completed to resembles observed populations. We follow the evolution of the obtained population during the collapse of the fragmented system, assessing the effect of initial stellar density on binaries.

\paragraph*{}
Finally, I present the various paths of research opened up by the \HubLem model: anisotropic expansion, mock observation with dust extinction, coupling with galactic-scale simulations. I conclude the thesis by recalling the initial context and summarising the results.

\paragraph*{}
Relevant material that is not necessary to follow the thesis is presented in various appendices. 