%%%%%%%%%%%%%%%%%%%%%%%%%%%%%%%%%%%%%%%%
%           Liste des packages         %
%%%%%%%%%%%%%%%%%%%%%%%%%%%%%%%%%%%%%%%%


%% Réglage des fontes et typo
%\usepackage[utf8]{inputenc}
%\usepackage[T1]{fontenc}
%\usepackage[frenchb]{babel}
%
%\usepackage{mathptmx}
%\usepackage{avant}



%% Apparence globale
\usepackage{color}
\usepackage[top=1.5cm, bottom=3.5cm, inner=2cm, outer=3cm]{geometry}

% \usepackage[french]{minitoc}		% Permet de faire une table des matieres par chapitre
% \setcounter{minitocdepth}{3}		% Mini-toc détaillées (sections/sous-sections)
% \usepackage{silence}
% \WarningFilter{minitoc(hints)}{W0023}	% Virer les erreur dues à minitoc
% \WarningFilter{minitoc(hints)}{W0024}
% \WarningFilter{minitoc(hints)}{W0028}
% \WarningFilter{minitoc(hints)}{W0030}
% \WarningFilter{blindtext}{} % this takes care of the `blindtext` messages

\usepackage{titlesec}
\usepackage{enumerate}
\usepackage{enumitem}
\usepackage{pdflscape}			% Permet d'utiliser des pages au format paysage
\usepackage{url}
\usepackage{listings}			% Permet d'insérer du code source
\usepackage{changepage}

\usepackage{natbib}
\usepackage{paralist}
\usepackage{aas_macros}

\usepackage{hyperref}
\definecolor{citations}{rgb}{0,0,0.4}
\definecolor{liens}{rgb}{0.5,0,0}
\hypersetup{
	bookmarks=false,
	unicode=true,
	pdffitwindow=true,
	pdfnewwindow=false,
	colorlinks=true,
	linkcolor=blue,
	citecolor=green,
	urlcolor=cyan}
% \usepackage{breakurl}
% \usepackage{ifmtarg}

%% Maths
\usepackage{amsmath}	% Permet de taper des formules mathématiques
\usepackage{amssymb}	% Permet d'utiliser des symboles mathématiques
\usepackage{amsfonts}	% Permet d'utiliser des polices mathématiques
\usepackage{amsthm}
\usepackage{nicefrac}	% Permet de taper de belles fractions
\usepackage{relsize}

\usepackage[ruled,vlined]{algorithm2e}
%\SetKwInput{Data}{\textbf{Données}}
%\SetKwInput{Res}{\textbf{Résultats}}
%\SetKwFor{Pour}{pour}{faire}{fin boucle}
%\SetKwBlock{Deb}{début}{fin}
%\SetKwInput{Entree}{Entrées}
%\SetKwInput{Sortie}{Sorties}
%%     \SetKw{KwA}{à}
%%     \SetKw{Retour}{retourner}
%%     \SetKwIF{Si}{SinonSi}{Sinon}{si}{alors}{sinon si}{alors}{finsi}
%%     \SetKwSwitch{Suivant}{Cas}{Autre}{suivant}{faire}{cas où}{autres cas}{fin d’alternative}
%%     \SetKwFor{Tq}{tant que}{faire}{fintq}
%%     \SetKwFor{PourCh}{pour chaque}{faire}{finprch}
%%     \SetKwFor{PourTous}{pour tous}{faire}{finprts}
%%     \SetKwRepeat{Repeter}{répèter}{jusqu’à}


%% Tableaux
\usepackage{array}
\usepackage{multirow}
\usepackage{booktabs}
\usepackage{colortbl}
\usepackage{threeparttable}	% Permet de faire des ``tablenotes'' dans un tableau
\newcolumntype{D}[1]{>{\centering}m{#1}}
\usepackage{diagbox}

%% Graphiques
\usepackage{graphicx}		% Permet l'inclusion d'images
\usepackage{pdfpages}		% Permet d'inclure des pdf plus facilement \includepdf
\usepackage{tikz}		% Permet de creer des elements graphiques
\usepackage{eso-pic}		% Permet d'ajouter des commandes graphiques a chaque page


%%%%%%%%%%%%%%%%%%%%%%%%%%%%%%%%%%%%%%%%
%         Commande Personnelles        %
%%%%%%%%%%%%%%%%%%%%%%%%%%%%%%%%%%%%%%%%
\titleformat{\chapter}[display]
  {\normalfont\Huge\bfseries\raggedleft}
  {\MakeUppercase{\chaptertitlename%}
      \enspace \thechapter}}
  {11pt}{\Huge}
%     \enspace \resizebox{!}{2cm}{\thechapter}
%     \enspace \rlap{\rule{5cm}{1.5cm}}}
% \titlespacing*{\chapter}{0pt}{30pt}{20pt}

\titleformat{\part}[frame]
  {\normalfont\bfseries\scshape}
  {\filright \Large \enspace Part \thepart \enspace}
  {11pt}{\Huge\filcenter}
% \titlespacing*{\part}{0pt}{30pt}{20pt}


% \newcommand{\partie}[1]{ \part{\textsc{#1}} }
\newcommand{\introchapitre}[1]{ \chapter*{\textbf{\MakeUppercase{#1}}}}
\newcommand{\biblio}[1]{ \chapter*{\textsc{#1}} }

%\newcommand{\algoref}[1]{Algorithme \ref{#1}}
\newcommand{\secref}[1]{Section \ref{#1}}
\newcommand{\chapref}[1]{Chapter \ref{#1}}
\newcommand{\figref}[1]{\textsc{Figure}. \ref{#1}}
\newcommand{\tabref}[1]{\textsc{Table}. \ref{#1}}

%\newcommand{\etco}{\textit{et coll.}\ }
%\newcommand{\dtitk}{\textit{DTI-TK}\ }
%\newcommand{\fa}{Fraction d'Anisotropie\ }
%\newcommand{\md}{Diffusion Moyenne\ }
%\newcommand{\da}{Diffusion Axiale\ }
%\newcommand{\dr}{Diffusion Radiale\ }
%\newcommand{\mlg}{Modèle Linéaire Général\ }
%\newcommand{\irmd}{Imagerie par Résonance Magnétique de diffusion\ }
%\newcommand{\rmn}{Résonance Magnétique Nucléaire\ }
%\newcommand{\itd}{Imagerie du Tenseur de Diffusion\ }

%\addto\captionsfrench{\def\tablename{\textsc{Tableau}}}
%
%\newcommand{\largedot}{\mathlarger{\mathlarger{\mathlarger{\cdot}}}}

%\makeatletter
%\def \@date{\data{today}}
%
%\def\@author{Author}
%\newcommand{\author}[2]{\def\@author{#1~\textsc{#2}}}
%
%\def\@title{Thesis title}
%\newcommand{\title}[1]{\def\@title{#1}}
%
%\def\@field{Field}
%\newcommand{\field}[1]{\def\@field{#1}}
%
%\def\@advisor{advisor}
%\newcommand{\advisor}[2]{\def\@advisor{#1~\textsc{#2}}}
%
%\def\@university{University}
%\newcommand{\university}[1]{\def\@university{#1}}

\makeatother


%%%%%%%%%%%%%%%%%%%%%%%%%%%%%%%%%%%%%%%%
%         En tete et pied de page      %
%%%%%%%%%%%%%%%%%%%%%%%%%%%%%%%%%%%%%%%%


\usepackage{fancyhdr}			% Permet de générer automatiquement des en-têtes et des pieds de pages

  \setlength{\headheight}{2cm}	% hauteur de l'en-tête
  
  %%%%%%%%% style front %%%%%%%%
  \fancypagestyle{front}
  {
      \fancyhf{}				% en-tête vide
      
      \fancyfoot[RO,LE]{\textbf{\thepage}}	% pied de page avec : page numéro
      \renewcommand{\footrulewidth}{0pt}
      \renewcommand{\headrulewidth}{0pt}
  }
  
   %%%%%%%%% style intro %%%%%%%%
  \fancypagestyle{introduction}
  {
      \fancyhf{}				% en-tête vide
      
      \fancyfoot[RO,LE]{\textbf{\thepage}}	% pied de page avec : page numéro
      \renewcommand{\footrulewidth}{1pt}
      \renewcommand{\headrulewidth}{1pt}
  }
  %%%%%%%%% style main %%%%%%%%
  \fancypagestyle{main}
  {
      \fancyhf{}
      
      \renewcommand{\chaptermark}[1]{\markboth{\textbf{\MakeUppercase{\chaptername}\ \thechapter.\ ##1}}{}}
%       \renewcommand{\sectionmark}[1]{\markright{\thesection\ ##1}}
      \renewcommand{\sectionmark}[1]{\markright{##1}}
      \fancyhead[L]{\leftmark\\[0.3em] \hspace{3cm} \rightmark \vspace{0.3em}}
      \renewcommand{\headrulewidth}{1pt}
      
      \renewcommand{\footrulewidth}{0pt}
      \fancyfoot[C]{\thepage}
  }
  
  %%%%%%%%% style back %%%%%%%%
  \fancypagestyle{back}
  {
      \fancyhf{}
      
      \fancyfoot[RO,LE]{\textbf{\thepage}}
      \renewcommand{\footrulewidth}{1pt}
  }
  
  
  