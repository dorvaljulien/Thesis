\chapter{Resum\'e en français}


\section{Introduction}
 Les \'etoiles naissent en groupe, lors de flamb\'ees de formation stellaire au sein de nuages mol\'eculaires. Les diff\'erentes \'etapes de cette formation stellaire sont pr\'esentes dans notre ciel: nuage mol\'eculaire froid, r\'egion d'\'emission HII peupl\'ee de coeur proto-stellaires, jeune amas enfoui (dans son gaz), jusqu'au stade \'evolu\'e d'un jeune amas d\'eposs\'ed\'e de son gaz primordial. Cette s\'equence \'evolutive de plusieurs millions d'ann\'ees se d\'eroule dans le ciel au fil des objets et r\'egions observ\'ees. 
 
La compr\'ehension de ce processus est cruciale pour appr\'ehender la formation stellaire en g\'en\'eral et même galactique. En effet, les amas globulaires sont consid\'er\'es comme des t\'emoins majeurs de la formation des galaxies. S\'epar\'es en populations bleues, à faible m\'etallicit\'e, et rouge à haute m\'etallicit\'e, les amas globulaires ont un lien complexe avec les fusions galactiques. Par exemple, des jeunes amas massifs, ou YMC (Young Massive Clusters), sont observ\'es dans des galaxies en train de fusionner, comme les Antennes. Ces objets sont consid\'er\'es comme de futurs amas globulaires.

Or ces même amas restent myst\'erieux. Longtemps consid\'er\'es comme les syst\`emes homog\`enes par excellence, un seul âge, une seule m\'etallicit\'e, plusieurs s\'equences stellaires ont r\'ecemment \'et\'e observ\'ees cohabitant au sein des mêmes amas. Ces populations multiples pourraient s'expliquer par le processus de formation des amas. Le gaz mol\'eculaire dont les \'etoiles \'emergent et les r\'egions de formations stellaires sont profond\'ement sous-structur\'es, les filaments et grumeaux sont la normes. Seuls les amas plus âg\'es ont une structure lisse, concentr\'ee et sym\'etrique. Cela implique une \'evolution dynamique, plusieurs grumeaux doivent fusionner pour former ces amas. Si ces grumeaux ont des m\'etallicit\'es et/ou des âges diff\'erents, cela pourrait expliquer les populations multiples des amas globulaires.

Les simulations hydrodynamiques permettent de reproduire la formation stellaire et d'observer l'\'evolution de ces sous-structures. H\'elas, elle sont souvent coûteuses en temps et en ressources de calcul, et sont limit\'ees à des petit syst\`emes (souvent un grumeau isol\'e), \'evoluant pour un temps assez court. Pour pouvoir aller plus loin et prendre en compte l’interaction entre les grumeaux, il est possible de prendre le relai avec des simulations à N-corps, en consid\'erant le gaz comme \'evacu\'e par les \'etoiles. Les simulations à N-corps sont moins gourmandes en ressources et permettent de mod\'eliser de bien plus grand syst\`emes sur des \'echelles de temps plus importantes.

Pour pouvoir simuler l'\'evolution de ces syst\`emes de mani\`ere r\'ealiste avec des simulations N-corps, il est n\'ecessaire de g\'en\'erer des conditions initiales r\'ealistes: sous-structur\'ees, sous-virielles comme le montrent plusieurs observations de jeunes amas. Plusieurs m\'ethodes existent pour g\'en\'erer ce genre de syst\`eme, mais elles se basent sur des simulations hydrodynamiques et restent coûteuse, ou produisent des syst\`emes avec une structure  artificielle dans l'espace des phases, ne prenant pas en compte l'\'etat dynamique du syst\`eme.

\paragraph*{}
Mon travail de th\`ese s'est concentr\'e sur la cr\'eation et l'exploration d'une nouvelle m\'ethode de g\'en\'eration de conditions initiales pour simuler de jeunes amas sous-structur\'es à travers des simulations N-corps. Cette m\'ethode est l'expansion de \HubLem, elle se base sur l'expansion radiale d'un syst\`eme uniform pour laisser des surdensit\'es naturelles se developper et construire des grumeaux auto-coh\'erents dans l'espace des phases.

Dans ce travail, j'approche d'abord la m\'ethode de mani\`ere analytique, d\'egageant les \'equations qui gouvernent l'expansion du syst\`eme. J'analyse ensuite des r\'ealisations num\'eriques du mod\`ele, en comparant les grumeaux aux observations et simulations hydrodynamiques. Je prend ensutie ce mod\`ele comme conditions initiales pour \'etudier la relaxation violente d'un amas jeune, sous-viriel et sous-structur\'e.

Dans un deuxi\`eme temps, je change d'\'echelle dynamique et m'int\'eresse aux \'etoiles binaires et à l'\'evolution de leur population dans ce genre de syst\`eme sous-structur\'es. J'analyse la population spontanée apparaissant dans les modèles de \HubLem, puis j'injecte des binaires supplémentaires pour me rapprocher des tendances des populations de binaires observées dans le champs galactique. Enfin, je laisse les modèles évoluer comme précédemment pour explorer l'influence des sous-structures et de l'effondrement sur la population de binaires.


\section{Partie I: Le modèle fragmenté et son évolution}


Pour générer un modèle fragmenté de \HubLem, on génère tout d'abord une sphère uniforme en tirant des étoiles d'une fonction de masse stellaire, Salpeter ou $L_3$. On attribue ensuite des vitesses radiales selon
\begin{equation}
\bold{v} = \Hub_0 \bold{r},
\end{equation}
analogue au champs de vitesse observé pour les galaxies dans l'univers, $\Hub_0$ étant l'équivalent de la constante de Hubble mais dont la valeur est ici libre. On laisse ensuite ce système évoluer à travers un intégrateur N-corps.

Pour mon travail, j'ai utilisé NBODY6, créé par Sverre Aarseth. Ce choix a été motivé par les nombreux avantages du code: c'est un intégrateur collisionel, capable de traiter des interaction proches et des étoiles binaires sans adoucir le potentiel, il possède de nombreuses optimisations algorithmiques qui font de lui un des intégrateurs N-corps les plus rapides, et enfin il possède une version GPU, permettant d'accélérer encore plus les simulations pour des nombres de particule suffisammen élevés.

Pendant l'expansion, les étoiles massives tendent à attirer des étoiles plus légère, faisant croitre les sur-densités initiales dans le modèle. On peut montrer par le calcul que la constante $\Hub_0$ doit être inférieure à $\sqrt{2}$ pour avoir un système lié et que l'expansion stoppe à un temps donné. Ce temps augmente rapidement lorsque qu'on se rapproche de cette valeur limite.

Il est possible de faire une analyse en perturbation du système en expansion en considérant des surdensités en coquille spheriques. Cela nous apprend que les surdensités devraient d'abord converger en régime linéaire, puis rentrer dans une phase d'évolution collisionelle, d'où peut émerger une ségrégation de masse.

\paragraph*{}
L'analyse des modèles fragmentés obtenus numériquement à travers NBODY6 nécessite de pouvoir isoler les grumeaux et les analyser. Pour cela, j'ai adapté la méthode utilisée par \cite{Maschberger2010} pour isoler les surdensités dans leur simulation hydrodynamique. on construit d'abord l'arbre couvrant de poid minimal, où MST (Minimum Spanning Tree) du système, puis en coupant toutes les branches plus longues qu'une certaine longueur, on considère tous les systèmes liés et isolés comme des grumeaux. J'ai fixé le seul paramètre libre, la longueur de coupure, en maximisant le nombre de grumeaux détectés. Ce choix donne des grumeaux cohérents, et un test a confirmé que l'algorithme permettait de retrouver une distribution théorique de grumeaux injectée dans un système.

L'exploration des différents paramètres et la réalisation d'une multitude de modèle de \HubLem a permis de dégager les caractéristiques suivantes: la fonction de masse des grumeaux est peu sensible au nombre total d'étoiles N ou à $\Hub_0$, mais elle est en revanche dépendante aux bornes de la fonction de masse stellaire. Une masse stellaire maximum à 100 $\Mo$ donne une queue de fonction de masse de grumeau en loi de puissance avec un index proche de -1, lorsque cette masse maximum descend à 20$\Mo$, la queue de la distribution accentue sa pente, l'index descend à -1.7. Le pic de la distribution se maintient à $\sim$ 20 $\Mo$.

Cela souligne l'importance majeure des étoiles massives dans la fragmentation. Cette importance est confirmée par les résultats suivants. Les grumeaux ont comparativement plus d'étoiles massives que les étoiles dites "du champs", celles n'étant pas détectées comme faisant partie d'un grumeau. la différence entre les distributions stellaires rappelle celle observées entre le champs galactique et les amas stellaires. De plus, la relation  $m_{max} - M_{grumeau}$ dans nos système recouvre la tendance observée dans des amas jeunes enfouis. Enfin, en utilisant la méthode de rang radial pour mesurer la ségrégation de masse, en accord avec \cite{Maschberger2010}, nous trouvons une tendance générale à la ségrégation dans nos grumeaux, similaire à ce qui est mesuré dans les sur-densités des simulations hydrodynamiques. Cette ségrégation se concentre sur les

\paragraph*{}
Ces résultats valident les modèles de \HubLem comme étant des conditions initiales adaptées pour simuler l'évolution dynamique de jeunes amas stellaires. Nous laissons ces modèles évoluer et subir une relaxation violente avant d'atteindre un état virialisé de quasi-équilibre. Pour pouvoir dégager l'influence des sous-structures sur cette évolution, nous avons également simulé la relaxation de modèles uniformes froids. Tous les modèles ont été simulés jusqu'à 40 unités de temps H\'enon, ce qui correspond ici à $\sim$ temps de traversée du système, mais bien moins qu'un temps de relaxation.

L'effondrement des modèles de \HubLem est plus doux que celui des systèmes uniformes en raison des grumeaux. En conséquence, le système central est moins densé. Les systèmes uniformes éjectent deux fois plus d'étoiles au rebond que les modèles fragmentés. Nous avons developpé une méthode d'extraction des étoiles éjectées permettant le retrait d'étoiles marginalement liées au système. En se concentrant sur les systèmes liées, le coeur plus concentré des systèmes uniformes accèlère leur évolution à deux-corps, ils s'étendent pour au final avoir une structure spatiale comparable aux systèmes fragmentés à t = 40 H.u.

En séparant les étoiles des systèmes par masse et en représentant l'évolution de leur répartion dans le système, on remarque que les systèmes uniformes tendant à developper une ségrégation de masse plus important que les systèmes fragmentés en fin de simulation. Pourtant, juste après l'effondrement, les modèle de \HubLem sont plus ségrégés, et cette ségrégation est concentrée sur les étoiles les plus massives. Cette ségrégation spécifique est héritée des grumeaux, dans lesquel le faible nombre d'étoiles pouvaient difficilement développer une ségrégation régulière, et se maintient dans l'évolution du système virialisé, en contraste avec la ségrégation des systèmes uniformes, plus étalée sur le spectre de masse stellaire. Dans un véritable amas stellaire, cela augmenterait les gradients de couleurs dans le coeur, en comparaison d'un système provenant d'un effondrement plus uniforme.



\section{Partie II: Les étoiles binaires dans les amas sous-structurés}

Dans la deuxième partie de cette thèse, je me suis concentré sur les étoiles binaires et le sort des populations de binaires dans les amas jeunes, sous-structurés et sous-viriels. Certains auteurs s'étaient déjà spécifiquement penchés sur ce problème \citep{Parker2011}, mais à travers des modèles fractaux, qui manquent la cohérence dynamique des systèmes de \HubLem, et pour des nombres d'étoiles limités à 1 ou 2 milliers. Le but de cette deuxième partie étaient d'explorer l'influence de la dynamique collisionelle à l’intérieur des grumeaux sur les systèmes binaires, ainsi que celle de l'effondrement et de son champs de marée lié au nombre total d'étoile.

J'ai développé un algorithme de détection d'étoile binaire dans les simulation à N-corps reposant sur une structure de donnée appelée arbre KD. La recherche de paire d'étoile liée est la première phase de l'algorithme, et cette recherche s'effectue, pour chaque étoile, parmi les voisins directs. L'arbre KD permet, une fois qu'il a été construit, d'effectuer des recherches de voisins $\log(N)/N$ plus rapidement qu'en force brute. Une fois que deux étoiles sont détectées comme liées, elles doivent définir une densité plus importante que celle créé par les voisins directs. Si c'est le cas, elles sont confirmées comme binaires. Cet algorithme a été testé et son paramètre libre validé en insérant une population de binaire connue dans un amas de King et en évaluant la solidité des systèmes retournés par l'algorithme.

L'application de l'algorithme aux systèmes fragmentés par la méthode de \HubLem a révélé l'existence d'une population de binaire spontanées, formées pendant l'expansion initiale, pendant laquelle les étoiles proches étaient fortement corrélées dans l'espace des phases. Cette population spontanée possède une fraction de binaire croissante avec la masse de la primaire. Cette tendance est similaire à la distribution observée dans le champs galactique, mais la population spontanée manque de systèmes avec des primaires de faible masse et des semi-axe majeur inférieurs à $\sim$ 1000 AU, qui sont pourtant majoritaires dans les populations observées. J'ai développé une méthode de complétion de la population. La nature particulière des modèles de \HubLem interdisait toute injection directe dans un système existant. Les binaires doivent être injectées dans la sphère uniforme initiale, en tant qu'objet unique portant la masse des deux composantes. Une fois l'expansion effectuée et le système fragmenté, les binaires peuvent être séparées et attribuée des positions et vitesses internes cohérents avec leur caractéristiques. La population injectée doit être tirée d'une distribution obtenue à travers un système d'équations non linaires qui modélisent la formation et la destruction de binaires au cours de l'expansion.

\paragraph*{}
Une fois ces modèles complétés, ils sont utilisés comme conditions initiales de la même manière que précédemment: effondrement, relaxation violente et virialisation, puis évolution dynamique plus régulière. Afin de dégager l'influence du nombre total d'étoiles, les modèles sont construit avec N=1.5k, 5k, 20k et 80k étoiles. Pour explorer l'influence de la densité , tous les modèles ont été construits avec deux densités stellaires différentes: 6 étoiles/$\pc^3$ et 400 étoiles/$\pc^3$. Ces valeurs sont tirés des observations de \cite{King2012,King2012a} et sont représentatives des extrêmes observés dans les régions de formation stellaires.

En mesurant la fraction total de binaires au cours du temps, on voit que les grumeaux détruisent 10 fois plus de binaires par unité de temps que le système virialié post-effondrement. Ces deux régimes sont d'autant plus clairs que le nombre total d'étoiles est élevé. Grâce aux puit de potential plus profond atteint par les amas avec 80k étoiles, ils détruisent deux fois plus de binaires que ceux avec 1.5k étoiles. La plupart des binaires avec des semi-axe majeur supérieur à 1000 AU sont détruites, alors que celles plus courtes que $\sim$ 100 AU sont très peu affectées. Une faible tendance est observées, les amas avec plus d'étoiles atteignent et détruisent des binaires légèrement plus sérrées, mais cette tendance est plus faible que ce qu'un raisonnement analytique pourrait prédire. Cette différence est probablement du aux sous-structures qui perturbent l'effondrement. Les binaires avec des primaires peu massives sont préférentiellement détruite dans les grumeaux, pendant l'effondrement. Dans le même temps, les primaires massive survivent mieux, voir voient leur population augmenter dans les modèles avec peu d'étoiles et un faible champs de marée. En revanche, une fois l'effondrement passé et le système virialisé, toutes les binaires sont affectées de la même manière par l'érosion de leur population.

\paragraph*{}
L'inspection détaillée des populations finales révèle l'existence des binaires "extremes": plus larges ou plus serrées que ce qui a été injecté dans le système. Les binaires très larges ont des semi-axes majeurs supérieurs à $10^4$ AU et se forment dans le halo d'étoiles éjectées lors du rebond. En effet, c'est un milieu à faible densité et aux vitesses corrélées, semblable à l'expansion initiale qui a vu naître les binaires spontanées.

De l'autre c\^oté de la distribution de semi-axe majeurs, 0.05\% des binaires détectées à al fin des simulations à faible densité stellaire étaient plus serrées que 0.6 AU, certaines avaient une séparation $\sim$ 0.01 AU. Ces systèmes n'ont pas été injecté dans les conditions initiales, et un suivi de leur évolution à montré qu'ils étaient le produit de collision binaires-binaires échangeant des étoiles. Paradoxalement, presque aucun système similaire n'a été trouvé dans les modèles à haute densité, alors que l'on pourrait attendre qu'une plus haute densité favoriserait les collisions et la formation de ces binaires. Nous avançons deux explications potentielles. Premièrement, ces binaires d'échanges sont rarement le produit d'échanges instantanés et forment souvent des systèmes à petit N, qui prennent un temps non négligeable pour véritablement procéder à l'échange, et une plus haute densité augmente de voir une autre étoile interrompre cet échange. Deuxièmement, des travaux analytiques et numériques sur les collisions et échanges impliquant des binaires semblent indiquer qu'une vitesse trop importante rend les collision aboutissant à des échanges plus difficiles. Une réponse définitive nécessitera un suivi particulier de ces systèmes pour véritablement comprendre leur origine.

\section{Perspectives et conclusion}

Dans les simulations présentées dans ce travail, nous avons fait l'hypothèse d'amas isolé, ce qui impliquait un effondrement isotrope et complet, permettant l'étude des processus de virialisation. Pourtant, il est important pour l'avenir du modèle de \HubLem d'inclure des champs de marée galactiques réalistes. Cela pourrait emp\^echer le collapse et extraire des grumeaux du système, tout en en fusionnant d'autres. Il serait intéressant de comparer la fonction de masse finale des grumeaux obtenus en présence d'un champs de marée avec celle des jeunes amas observés dans le m\^eme intervalle de masse. Des champs de marée tirés de simulations à échelle galactique pourrait même être appliqués à nos modèles gr\^ace au code Nbody6tt, developpé par \cite{Renaud2011}.

Une autre hypothèse qui traverse ce travail est que l'évolution stellaire des étoiles les plus massives du système n'affecterait pas la dynamique générale de manière sensible. Cela est vrai pour les intervalles de masse stellaire que nous avons choisi, mais pour se rapprocher des observations, il est nécessaire d'inclure des étoiles plus massives. Il est alors indispensable de prendre leur évolution et perte de masse en compte car cela va probablement perturber la dynamique interne des grumeaux, leur fonction de masse et surtout l'évolution des binaires très courtes que nous avons détectées, qui seront des binaires de contacts.

Une autre voie de recherche potentielle est l'expansion anisotrope: en remplaçant la constante scalaire $\Hub_0$ par une matrice, il est possible de favoriser l'expansion le long d'un axe, ainsi que d'introduire une rotation, afin d'injecter un moment angulaire non nul dans le système. On peut également imaginer l'addition de gaz dans le système pour obtenir un système fragmenté permettant l'étude du couplage gaz-étoile dans les jeunes amas.

Enfin, les modèles de \HubLem permettent la génération d'observations artificielles pour analyser l'influence de la ségrégation de masse sur la morphologie de jeunes amas. En utilisant des modèles stellaires pour obtenir les luminosités de nos étoiles, puis en applicant une extinction réaliste due à la poussière, il est possible d'explorer à quel point la ségrégation de masse et les grumeaux obtenus dans les observations sont dépendants de la limite de détection.


\paragraph*{}
Pour conclure, la méthode de \HubLem est prometteuse et a déjà produit des résultats intéressants sur l'évolution des jeunes amas et de leurs populations de binaires. L'idée de départ est simple, mais le résultat recouvre suffisamment de résultats numériques et observationnels sur la formation stellaire pour que le modèle puisse être considéré comme un bon point de départ pour simuler des amas sous-structurés.

La ségrégation de masse présente dans les grumeaux se transmet au système virialisé, qui présente alors un indice de son origine fragmentée, comparé à un système initialement uniforme. Quand aux populations de binaires, nous avons montré que les grumeaux étaient plus efficaces qu'un système central pour éroder la population de binaires, tout en repérant la formation de binaires extrêmes.

Il existe de nombreuses voies de recherches prometteuse pour le modèle de \HubLem, toute permettant de s'approcher un peu plus de la réalité des amas jeunes et de pouvoir simuler leur formation de manière rapide et efficace.
