%%%%%%%%%%%%%%%%%%%%%%%%%%%%%%%%%%%%%%%%%%%%
% Chapitre 3 - INTRO
%%%%%%%%%%%%%%%%%%%%%%%%%%%%%%%%%%%%%%%%%%%%

\chapter*{Introduction}
\markboth{\textbf{Introduction}}{}
%---------------------------------------------------------------------------------------

Cette partie aborde les diverses problématiques liées à la détection de changements ainsi que les solutions proposées au cours de la thèse. 
Vu la complexité et l'importance d'un sujet tel que la détection de changements en imagerie médicale, les notions présentées seront spécifiques à l'imagerie du tenseur de diffusion. 
Pour un état de l'art plus général, il est conseillé de se réferrer à {\color{red}CITATIONS}.


Le point de départ, ou plus présicément la problématique centrale, peut se résumer de la sorte :
la modéle mathématique de la diffusion prend la forme d'un tenseur de diffusion d'ordre 2. Ce tenseur est caractérisé par une matrice symétrique définie positive de dimension 3 (\chaptername\ \ref{Chapter1}) dont les six éléments supérieurs décrivent totalement la diffusion de l'eau en chaque voxel de l'image. 
Comment prendre en compte toute l'information de diffusion (contenue dans ces six éléments) pour détecter des changements en imagerie de diffusion ?


Le premier chapitre introduit la notion principale de cette thèse : la comparaison de groupes. 
Un contexte et les différentes problématiques sont exposés ainsi qu'une revue ciblée des différentes méthodes déjà existantes dans la litérature.
Enfin, les pré-traitements, répondant à certaines premières problématiques et retenus pour la chaîne de traitements, sont décrits.


Le deuxième chapitre présente la solution développée au cours de cette thèse pour répondre à la problématique centrale. 
Elle se base sur le Modèle Linéaire Général (MLG) et propose de l'étendre, sous certaines hypothèses, aux tenseurs de diffusion.
En effet, ce modèle est le plus répandu en comparaison de groupes en ITD. 
Il est notamment utilisé par le logiciel \og Statistical Parametric Mapping \fg (SPM\footnote{\url{http://www.fil.ion.ucl.ac.uk/spm/}}).


Le \chaptername\ \ref{Chapter1} de la première partie aborde également les bases sur la géométrie particulière des tenseurs de diffusion : la géométrie Riemanienne.
En effet, les métriques Euclidennes ne permettent pas de tenir compte de la positivité des matrices tensorielles.
Pour pallier à cette limite, différentes approches ont été proposées comme la géométrie Log-Euclidenne \cite{Arsigny2006} ou encore des métriques Riemaniennes.
Ces points sont traités dans le troisième chapitre de cette partie.

Dans le quatrième chapitre, nous verrons quels post-traitements généraux sont appliqués et comment pousser encore plus loin les études de comparaison de groupes en imagerie du tenseur de diffusion.
